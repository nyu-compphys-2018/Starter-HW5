\documentclass{article}
\usepackage{listings,amsmath}

%Listings Settings
\lstset{frame=tb,
language=python,
aboveskip=3mm,
belowskip=3mm,
showstringspaces=false,
columns=flexible,
basicstyle={\ttfamily}
}

%Custom Commands
\newcommand{\git}{{\texttt{git}}}
\newcommand{\github}{{\texttt{Github}}}
\newcommand{\Python}{{\texttt{Python}}}
\newcommand{\python}{{\texttt{python}}}
\newcommand{\Anaconda}{{\texttt{Anaconda}}}
\newcommand*\diff{\mathop{}\!\mathrm{d}}
\newcommand*\Diff[1]{\mathop{}\!\mathrm{d^#1}}
\newcommand{\ket}[1]{\left |#1\right \rangle}  
\newcommand{\bra}[1]{\left \langle #1\right |}  
\newcommand{\braket}[2]{\left \langle #1 \vert #2 \right \rangle}

\begin{document}

\begin{center}

\vspace*{-2.5cm}
\LARGE
\bf{Homework 5}
\vspace{1cm}

\large{Due: Oct. 26, 2018}
\vspace{1cm}

\end{center}

Write a \Python{} program to study orbits by solving ordinary differential equations with a variety of methods. Validate your methods and demonstrate their convergence using the Kepler problem. Apply your methods to study the motion of Halley's comet and the star S2.

\begin{enumerate}
	\item {\bf Convergence} 
		Write \Python{} functions to evolve a system of first-order ODEs $\dot{x}_i(t) = f_i(t, x)$ using Forward-Euler, RK2, RK4, and Verlet.  For each method, using a variety of time steps, compute the orbit of Mars for $T=5$ (Earth) years, beginning from aphelion.  Ignore the effect of other planets, and assume the sun is fixed.  Compute the exact position $\vec{r}(T)$ of Mars at this time by solving the Kepler problem. Make a convergence plot of the $L_1$ error $|\vec{r}(T)_{numerical} - \vec{r}(T)_{exact}|$ versus number of steps $N$ for each method.  Plot the energy versus time for each method as well.  
		\begin{itemize}
			\item Do the methods converge as expected?  How many time steps per orbit are required for the theoretical scaling to begin?
			\item How well do the methods conserve energy?  When may this be a problem?
		\end{itemize}
	\item {\bf Halley's Comet} 
		Write an \emph{adaptive} RK4 method, and use it to compute the orbit of Halley's comet for a few orbital periods. Compare its performance to non-adaptive RK4 and Verlet.  How long does it take to run each and get similar errors?
	\item {\bf S2} 
		The star S2 orbits the black hole at the centre of the Milky Way, Sagitarrius A*, sufficiently close that general relativity is important.  One can treat general relativity in an approximate way by replacing the Newtonian potential $\phi_N = -GM/r$ with the Paczy\'{n}ski-Wiita potential $\phi_{PW} = -GM / (r-r_S)$, where $r_S = 2GM/c^2$ is the Schwarzschild radius of the mass.  Compute several orbits of S2 using both $\phi_N$ and $\phi_{PW}$.  What is the difference?  Using your results, calculate the precession of the periapse of S2.  How many time steps per orbit should you use to ensure an accurate answer?  Earth-based telescopes in the coming years are going to measure this!
\end{enumerate}

Write a report summarizing your work, showing all plots, giving your results, and discussing the questions.  Include the \texttt{.pdf} version of the report and all Python files in the repo.

\vspace{1cm}

\textbf{Reference:} See Newman Chapter 8 for discussion of ODE solvers and Chapter 6 for methods of solving non-linear equations (like the kepler problem!).

\end{document}

